\chapter*{Введение}                         % Заголовок
\addcontentsline{toc}{chapter}{Введение}    % Добавляем его в оглавление

\paragraph*{Актуальность темы.}

\textbf{Сырая природный газ, добытый из нефтяных скважин, должен быть подвергнут обработке и переработке на газовом заводе, чтобы соответствовать стандартам качества, установленным газопроводными компаниями. Однако, когда питающий газ содержит значительное количество газовых конденсатов (NGL), возникают экономические стимулы для их извлечения, которые обычно могут приносить большую стоимость, чем природный газ с тем же теплотворным эффектом. На рисунке 1 приведена блок-схема газоперерабатывающего завода, предназначенного для производства NGL. Извлеченный поток NGL обычно направляется на фракционирование для дальнейшей переработки в отдельные продукты. Компонент этана может быть продан в качестве сырья для нефтехимических заводов. Пропановые и бутановые жидкости оцениваются как жидкое топливо. Сжиженный газ (LPG), состоящий из пропана и бутана, широко используется как альтернативное топливо для домашних нужд, а также как сырье для химической промышленности. Природный бензин (конденсат C5+) может быть экспортирован в нефтеперерабатывающие заводы в качестве компонента для смешивания с моторным бензином. Гидроуглеводные жидкие продукты из установки фракционирования NGL иногда не соответствуют спецификациям заказчика (см. Таблицу 1). Фактически, если в питательном газе присутствуют кислотные и содержащие серу соединения и они не были удалены до извлечения NGL, они окажутся в продуктах NGL (см. Таблицу 2). Эти жидкие продукты могут потребовать дополнительной обработки для удаления этих загрязнителей. Однако, если эти загрязнители в значительной степени удаляются на этапе подготовки сырья, дальнейшая обработка может быть сокращена или исключена. Существуют две основные причины удаления кислотных газов и соединений серы из углеводородных жидкостей: (1) защита окружающей среды путем уменьшения или исключения количества токсичного сероводорода (H2S) и/или диоксида серы (SO2), высвобождающегося или образующегося во время сгорания, и (2) защита оборудования процесса, контактирующего с кислыми потоками углеводородных жидкостей. Фактически, эти загрязнители могут вызывать коррозию оборудования, если жидкость недостаточно обезвожена. Сероводород достаточно коррозионно активен при концентрациях более 0,55 ppmw, что может привести к провалу испытания на медную полосу для LPG, в то время как 2 ppmw элементарной серы могут вызвать провал испытания на медную полосу (Pyburn et al., 1978). Если присутствуют как сероводород, так и элементарная сера, пороговые значения для провала испытания на медную полосу значительно снижаются (там же). Меркаптаны (RSH) нежелательны в углеводородных жидких продуктах прежде всего из-за запаха. До 100 ppm меркаптанов не приведут к провалу испытания на медную полосу и даже могут обеспечить некоторое торможение реакции H2S с медной полосой (там же). Однако, если также присутствует элементарная сера, меркаптаны вызовут провал испытания на медную полосу.Карбонилсульфид (COS) и дисульфид углерода (CS2), хотя и не коррозионно активны в LPG, медленно гидролизуются до H2S в присутствии свободной воды, что приводит к выходу продукции из спецификации (Nielsen et al., 1997). Если COS не удаляется из потока сырья, он концентрируется в потоке пропана фракционатора. Затем COS может быть обработан с помощью твердых адсорбентов, либо регенерируемых, либо нерегенерируемых (Amiridis, 2006). Наконец, присутствие значительного количества диоксида углерода (CO2) может увеличить паровое давление и снизить теплотворную способность углеводородных жидкостей (Mokhatab et al., 2015). Обратите внимание, что стабилизированный конденсат должен быть подвергнут обработке для удаления тяжелых меркаптанов и других нежелательных загрязнителей до очень низких уровней, чтобы получить жидкий продукт, который соответствует спецификациям для продажи как "природный бензин". В то время как паровая перегонка в стабилизаторе может быть использована для удаления легких углеводородных и кислых газовых компонентов, она имеет минимальное воздействие на удаление тяжелых меркаптанов (Mokhatab et al., 2015). Если конденсат содержит меркаптаны более низкой молекулярной массы (например, метилмеркаптан), его можно обработать с помощью традиционных технологий жидкостной обработки, таких как щелочная промывка, процесс Merox™ от UOP, процесс THIOLEX™ от Merichem, молекулярные сита или твердые катализаторные слои. Если конденсат содержит меркаптаны более высокой молекулярной массы, ароматические сернистые соединения (например, тиофен) и другие нежелательные серные компоненты, его необходимо обрабатывать с помощью гидроочистки, которая является распространенным процессом в нефтеперерабатывающих заводах для десульфурации сырья с высоким содержанием серы (Mokhatab et al., 2015).}

\paragraph*{Цель работы.}

\paragraph*{Задачи работы.}

\paragraph*{Научная новизна работы.}

\paragraph*{Теоретическая и практическая значимость работы.}

\paragraph*{Положения выносимые на защиту.}
\begin{enumerate}
    \item \statementOneRU
    \item \statementTwoRU 
\end{enumerate}

\paragraph*{Апробация работы.}

\paragraph*{Достоверность научных достижений.}

\paragraph*{Внедрение результатов работы.}

\paragraph*{Публикации.} Список всех публикаций автора по теме диссертации:
\begin{refsection}[biblio/own.bib]
\nocite{*}
\printbibliography[
    keyword=own,
    heading=none,
    resetnumbers=true
]
\end{refsection}



\paragraph*{Структура и объем диссертации. }
Диссертация состоит из~введения,
\formbytotal{totalchapter}{глав}{ы}{}{},
заключения и
\formbytotal{totalappendix}{приложен}{ия}{ий}{}.
Полный объём диссертации составляет
\formbytotal{TotPages}{страниц}{у}{ы}{}, включая
\formbytotal{totalcount@figure}{рисун}{ок}{ка}{ков} и
\formbytotal{totalcount@table}{таблиц}{у}{ы}{}.
Список литературы содержит
\formbytotal{citenum}{наименован}{ие}{ия}{ий}.